\documentclass[9pt]{article}

\usepackage{amsmath}
\usepackage{epsfig}

\begin{document}

\title{{\sc BTKernel} Manual}
%\subtitle{Subtitle Text, if any}
\author{}
%\authorinfo{}
%           {}
%           {}

%\markboth{DRAFT: DO NOT REDISTRIBUTE}{DRAFT: DO NOT REDISTRIBUTE}
\maketitle

\begin{abstract}
This manual describes the usage and the code structure of our
Linux module which performs fast dynamic binary translation for the
kernel \cite{sosp13}.
\end{abstract}

\section{Introduction}
\label{sec:intro}
This document is a manual for the Linux kernel module, {\sc BTKernel}, as
described in \cite{sosp13}.
For information on running the module, please see the README file packaged
with this distribution in the top-level directory.

Section~\ref{sec:peeptab} describes the peephole translation table
used to specify the translation rulebook. Section~\ref{sec:generic_rules}
specifies, how to add generic rules to the translation rulebook, that do
not depend on a particular opcode. Section~\ref{sec:implementation} provides
information about the code structure of our binary translation
module.

\section{Peephole Table ({\em aka} Translation Rulebook)}
\label{sec:peeptab}
The translation rulebook (as described in \cite{sosp13}) is stored a
file called {\tt peep.tab}. The rulebook is a collection of {\em peephole table
entries} (or {\em rules}), marked by the ``{\tt entry:}'' keyword in the file. Each entry
has three sections, each separated by a line containing only ``{\tt --}''
characters. The entry itself is terminated using the ``{\tt ==}'' characters.

The first section specifies the {\em source
instruction sequence}, which is a pattern of a native instruction sequence that
should be matched in the native instruction stream, for this rule to be applied.
The last section specifies the {\em destination instruction sequence}, or
the sequence that should be emitted in place of the source sequence, if the
pattern matches. The optional middle section specifies constraints on the
variables used in the source and/or destination sequence, if any. We provide
examples below to clarify this.

The following rule
\begin{verbatim}
entry:
  pushl %ds
  --
  pushfl
  pushl %eax
  movl %fs, %eax
  cmpl $(__KERNEL_PERCPU), %eax
  popl %eax
  jne 1f
  popfl
  pushl %fs:tx_target
  jmp 2f
  1: popfl
  pushl %ds
  2: nop
  ==
\end{verbatim}
specifies that if the native instruction stream has an instruction
``{\tt pushl \%ds}'', then it should be replaced by the 12 instructions
given in the following section of this rule. In this case, the destination
sequence first saves the flags (using {\tt pushfl}), then saves {\tt eax}
to allow its use as a temporary, compares {\tt \%fs} with a
constant ({\tt \_\_KERNEL\_PERCPU}, which indicates that the
interrupt/exception was received while executing in kernel mode), and
if found equal, pushes
{\tt txtarget} instead. If found
unequal (indicating that the interrupt/exception was received
in user mode), it performs the original
{\tt ``pushl \%ds''}. This logic is used to save the special memory
location {\tt tx-nextpc-loc}, as described in \cite{sosp13} ({\tt tx-nextpc-loc}
is called {\tt tx\_target} in the code).

Notice that the peephole rule can refer to kernel
constants (like {\tt \_\_KERNEL\_PERCPU}). Also notice that it can
refer to 
the special variable {\tt tx\_target}).

In our translation rules, we use variables of the
type {\tt \%vrNN} (where NN is a number)
to refer to register names in the source instruction sequence. We call
these ``source register variables'' (or source registers, for short). 
A source register variable can be suffixed with a '{\tt b}', '{\tt w}', or
'{\tt d}' character to represent 8-bit, 16-bit, and 32-bit incarnations
of that registern name. For example, if {\tt \%vr0d} represents
register {\% eax}, then {\tt \%vr0w} represents {\tt \%ax}, and
{\tt \%vr0b} represents {\tt \%al}. The higher 8-bit registers are
represented using the capital '{\tt B}' suffix. In the example above,
{\tt \%vr0B} will represent register {\tt \%ah}. Here is an
example of a rule involving a source register variable:
\begin{verbatim}
entry:
  add $1, %vr0d
  --
  inc %vr0d
  ==
\end{verbatim}
This rule specifies that if the source instruction stream has an
instruction which adds the constant ``\$1'' to a register ({\em any}
register), then that instruction must be replaced with an
increment ({\tt inc}) instruction operating on the same register.
This single rule will ensure that all:
\begin{itemize}
\item ``{\tt add \$1, \%eax}'' get translated to {\tt inc \$1, \%eax}.
\item ``{\tt add \$1, \%ecx}'' get translated to {\tt inc \$1, \%ecx}.
\item \ldots
\item ``{\tt add \$1, \%edi}'' get translated to {\tt inc \$1, \%edi}.
\end{itemize}
Thus {\tt \%vr0d} represents a register variable that is matched with
a real register in the source instructions, and is replaced by its
matched value in the translated instructions. On 32-bit
x86, an input rule can have
up to six register variables {\tt \%vr0 \ldots \%vr5}.

The translated sequence may need to use temporary registers, i.e., registers
that are not used in the source sequence but are needed to store
temporary values in the translation. We allow temporary register variables
of the type {\tt \%trNN} (where NN is a number) to refer to
temporary register names. These temporary registers are chosen at
translation time, and are appropriately saved and restored, before and
after the translated code respectively. As discussed in \cite{sosp13},
the save and restore is implemented using {\tt push/pop} instructions in
our translator.
Here is an example of a rule involving temporary register variables:
\begin{verbatim}
entry:
  xorl %vr0d, %vr1d
  xorl %vr1d, %vr0d
  xorl %vr0d, %vr1d
  --
  movl %vr0d, %tr0d
  movl %vr1d, %vr0d
  movl %tr0d, %vr1d
  ==
\end{verbatim}
In this rule, both the source and the translated sequences exchange the
values of {\tt vr0d} and {\tt vr1d}. This rule specifies that if a source
pattern is seen, replace it with a target pattern. Notice that the target
pattern uses a temporary register ({\tt tr0d}). Our translator automatically
identifies a register name to be used as a temporary, at translation time.
This temporary register is chosen carefully so that it does not conflict with
another register in the source sequence (e.g., {\tt vr0}) in an incorrect
way.
Also, a block-level liveness analysis is
performed to intelligently choose the temporary registers at translation time.
If the chosen temporary register is live, it is saved and restored, before
and after the translated code respectively.

Here is an example involving register constraints:
\begin{verbatim}
entry:
  jmp *%vr0d
  --
  %tr0d: no_esp
  %vr0d: no_esp
  --
  pushfl
  cli
	pushl %tr0d
  movl %vr0d, $tr0d
  andl $JUMPTABLE_MASK, %tr0d
  cmpl %tr0d, %fs:jumptable(,%tr0d,8)
  jne 1f

  #jumptable hit
  movl %fs:jumptable+4(,%tr0d,8), %tr0d
  movl %tr0d, %fs:txnextpc
  popl %tr0d
  popfl
  jmp %fs:txnextpc

  #jumptable miss
  1: movl %vr0d, %fs:nextpc
  popl %tr0d
  jmp dispatcher_entry
  # dispatcher will call popfl before jumping back to code cache
  ==
\end{verbatim}
This rule specifies that an indirect branch using a register to
compute the target should be translated to instructions that index
the jumptable, and jump to the translated target/dispatcher on a
jumptable hit/miss respectively. Notice that this rule employs one
temporary register, {\tt \%tr0d}. Further, this rule has
a middle section which specifies the constraints on the register
variables. In this case, the constraints specify that
the rule can be applied only if {\tt \%vr0d}
is not matched to the register {\tt \%esp} (stack pointer), i.e., the
source instruction should not be of the form {\tt ``jmp *\%vr0d''}. Also,
the chosen temporary register should be different from {\tt \%tr0d}. Notice
that these constraints are needed because the translation modifies
{\tt \%esp} through {\tt push} and {\tt pop} instructions.
In this rule, {\tt nextpc} and {\tt txnextpc} refer to the native
and translated PCs that need to be executed next, as also discussed
in \cite{sosp13}.

The pattern matching rules additionally allow the specification of
{\em constant variables}, denoted by {\tt C0}, {\tt C1}, {\tt C2}, \ldots.
If a constant variable is specified in the source instruction sequence,
it can match with any constant value. The value with which it is matched
will be replaced in the translated sequence at each occurrence of that
constant variable. Here is an example of a rule involving a constant
variable:
\begin{verbatim}
entry:
  movl $C0, %vr0d
  movl %vr0d, %vr1d
  --
  movl $C0, %vr0d
  movl $C0, %vr1d
  ==
\end{verbatim}
This rule specifies that if we find a pattern in the instruction sequence
where a constant is first moved to a register and then that register is
copied to a second register; then, it should be replaced by an instruction
sequence that writes the same constant in both registers directly. This
rule will translate ``{\tt mov \$234, \%eax; mov \%eax, \%ecx}'' to
``{\tt mov \$234, \%eax; mov \$234, \%ecx}''. Notice that the constant
{\tt 234} is matched with the constant variable {\tt C0} in this case.

The {\tt peep.tab} file is given to a C preprocessor, before parsing it
for our use. Hence, it supports C preprocessor macros (using {\tt \#define},
{\tt \#include}, etc.).

\subsection{Generic Translations}
For some applications, we need generic rules, which are hard to specify
as pattern matching rules using the patterns described above.
Some examples of rules of this type are:
\begin{itemize}
\item For each memory access, do something (e.g., write to corresponding
shadow byte).
\item For each memory access through a particular segment register, do
something (e.g., intercept all accesses to code segment {\tt \%cs}).
\item For each write to a particular register (e.g., stack pointer {\tt \%esp}),
do something.
\item For every instruction, do something (e.g., count the number of
instructions executed).
\end{itemize}

Because our patterns require specification of the instruction opcode, such
generic rules will require several pattern-matching rules (one for each
opcode) to be expressed in our system. To alleviate this problem, we
provide an alternate mechanism to specify such translation rules, through
the special function {\tt peepgen\_emit\_code()} to generate translated code:
\begin{verbatim}
void
peepgen_emit_code(char *obuf, char *ebuf, int label, int n_args, ...);
\end{verbatim}
This function acts as a code generator, and fills-in arrays
{\tt obuf} and {\tt ebuf}
with the generated code and the ``edge code'' respectively
(see \cite{sosp2013} on why different arrays are needed for translated and
edge codes). The third argument to the code generator is a {\tt label} which
specifies the code pattern that should be used to
generate this code, and the fourth and subsequent arguments
specify the parameters which may be required by this code pattern to generate
the final code. We make this clearer with the following example of
the {\tt inscount} instrumentation client, which
counts the total number of instructions executed. This client is also
described in \cite{sosp13}.

At translation time, we call the function {\tt count\_insns()} (defined
in {\tt peep/instrument.c}) at the beginning of every basic block. The
{\tt count\_insns()} function is declared as follows:
\begin{verbatim}
void
count_insns(char *optr, char *eptr, int num_insns);
\end{verbatim}
This function generates code to add the value in {\tt num\_insns} (determined
at translation time) to the global variable that maintains the count of the
total number of instructions at runtime. The function is defined as
follows:
\begin{verbatim}
void count_insns(char *optr, char *eptr, int num_insns)
{
  peepgen_emit_code(optr, eptr, peep_snippet_inc_sreg0, 1, num_insns);
}
\end{verbatim}
The call to {\tt peepgen\_emit\_code()} takes as argument the label
{\tt peep\_snippet\_inc\_sreg0}. This label identifies the code represented
by the macro {\tt INC\_SREG0} defined in {\tt peep/peeptab\_defs.h}. Notice
that the macro name is identified by stripping the {\tt peep\_snippet\_}
prefix from the label name, and capitalizing the rest of the letters.

The {\tt INC\_SREG0} macro is defined in {\tt peep/peeptab\_defs.h} as:
\begin{verbatim}
#define INC_SREG0 \
  pushfl; \
  addl $C0, %fs:totalcount; \
  adcl $0, %fs:(totalcount+4); \
  popfl
\end{verbatim}
This assembly fragment has a similar structure as the last section of a
pattern-matching rule (as described in Section~\ref{sec:peeptab}). It
first saves the flags, adds {\tt C0} to the 64-bit global per-CPU variable
called {\tt totalcount} and then restores the flags to their original value.
Notice that this assembly fragment uses a constant variable, {\tt C0}.
The concrete value of this constant is determined at translation time; in
this case, the value of the constant is specified as the {\tt num\_insns}
argument variable in {\tt count\_insns}. Thus, the {\tt count\_insns} function
will generate code of the form:
\begin{verbatim}
pushfl
addl $<num_insns>, %fs:totalcount
adcl $0, %fs:(total_count + 4)
popfl
\end{verbatim}
at the beginning of a basic block, where {\tt <num\_insns>} is substituted
by the value of the last argument to {\tt count\_insns()}. Notice that the
value of this argument will be different for each basic block.

We next discuss our implementation of shadow memory briefly. The function
{\tt instrument\_memory()} in {\tt peep/instrument.c} instruments
each memory access to update the shadow bytes appropriately. We first
check if the instruction accesses non-stack memory. If so, we generate
code to push flags and the temporary register, and
call {\tt gen\_shadow\_code()} which generates code to update the
respective shadow bytes. This function takes as argument, the disassembled
instruction structure {\tt insn}, and its memory operand {\tt op1} which
needs to be instrumented. The {\tt gen\_shadow\_code()}, in turn, calls
{\tt peepgen\_emit\_code()} with proper labels to generate the desired
code for updating shadow bytes.

\section{Implementation}
\label{sec:implementation}
Here, we discuss the implementation structure of our binary translation
module. The user-level executables for loading and unloading the module
are in the top-level {\tt btmod/} directory. The {\tt launch\_bt} and
{\tt terminate\_bt} programs
(sources in {\tt launch\_bt.c} and {\tt terminate\_bt.c}) launch and
terminate the DBT module respectively. Their operations are described
in \cite{sosp2013}. Essentially, these programs call open
on {\tt /dev/btkernel} and execute {\tt ioctl} calls on it to
configure and start/stop translation.

The top-level module functions are implemented in {\tt bt/btmod.c}
and {\tt bt/btmain.c}. We
only discuss the {\tt create\_shadow\_idt()} function, which creates
a shadow IDT pointing to code cache addresses corresponding to the original
native code addresses.

The {\tt create\_shadow\_idt()} function iterates over the IDT entries
(see call to {\tt bt\_read\_idt\_addr()}) and replaces them with the
translated addresses (see call to {\tt bt\_put\_idt\_addr()}). The
translated code cache addresses are obtained using the function
{\tt tb\_gen\_translated\_code\_idt()}, which eventually calls
{\tt \_\_tb\_gen\_translate\_code()}. This function allocates
a translation block (also called {\em code block} in \cite{sosp13}),
initializes its starting PC value ({\tt eip\_virt}), and calls
{\tt translate()} on it.
The {\tt translate()} function is called twice, first to determine the
size of the translation, and then to perform the actual translation.
After the first call to {\tt translate()}, {\tt tb\_malloc()} is called to
allocate appropriate space for the translation block, based on the
computed size of the translation. Then the second call to {\tt translate()}
actually produces the translated code in the buffers allocated by
{\tt tb\_malloc()}.

The {\tt translate()} function performs the actual translation by
looking at each instruction, and calling {\tt peep\_translate()} on
it\footnote{While the peephole rules allow patterns with multiple
source instructions, this translator currently only supports
single instruction translations. Multiple source instruction translation
has previously been used for \cite{asplos06, osdi08}.}. The
{\tt peep\_translate()} function searches the table of peephole rules, which
is organized as a hash table for fast lookup, to find a rule that
matches. A match is determined by the {\tt templ\_matches\_insns()} function.
If the match is successful, the function fills-in the {\tt assignments}
argument with the mappings from variables to
constants (e.g., {\tt vr0 -> eax}, {\tt C0 -> 234}, etc.). If the
peephole rule uses temporaries, temporary registers are determined.

The {\tt nomatch\_pairs} field in a peephole entry determines which register
variables cannot be matched to the same register name. For example, if
{\tt (vr0, tr0)} belongs to the set of {\tt nomatch\_pairs}, {\tt vr0}
and {\tt tr0} must be mapped to different registers. Similarly,
if {\tt (vr0, vr1)} belongs to the set of {\tt nomatch\_pairs}, the
input sequence cannot match {\tt vr0} and {\tt vr1} to the same register.
The {\tt nomatch\_pairs} for an entry are automatically computed using
a use-def analysis on the translated code at table initialization time.

Finally the {\tt peepgen\_code()} function is called with arguments
specifying the {\tt label} and its parameters. This label is similar to
the one used for emitting generic translations in
Section~\ref{sec:generic_translations}. At compilation time,
these labels are emitted and associated with the assembly code fragments
specified in their peephole rule. The parameters to this label (passed
to {\tt peepgen\_code()} in the {\tt peep\_param\_buf} argument) are
obtained from the {\tt assignments} variable, which was filled-in by
the {\tt templ\_matches\_insns()} function. We explain this with an example
below.

Consider the following peephole rule.
\begin{verbatim}
entry:
  movl $C0, %vr0d
  movl %vr0d, %vr1d
  --
  movl $C0, %vr0d
  movl $C0, %vr1d
  ==
\end{verbatim}
Assume that the source instruction stream contains a sequence
of the form ``{\tt movl \$123, \%esi; movl \%esi, \%edi}''. This
sequence will get matched to the source instruction sequence template
of this rule, and so the {\tt templ\_matches\_insns()} functions
will return true, with the {\tt assignments} variable containing the
mappings:
\begin{verbatim}
  C0 --> 123
  vr0 --> esi
  vr1 --> edi
\end{verbatim}
The constant values {\tt 123}, {\tt esi}, and {\tt edi}, will become the
parameters for the translated assembly fragment. The translated assembly
fragment will be associated with a {\tt label}; in our implementation, we
name a label by the line number at which the peephole rule was encountered
in the {\tt peep.tab} file. Assume that this translated assembly fragment
is associated with label, {\tt rule450}. In that case, {\tt peepgen\_code()}
function will receive as arguments {\tt rule450} and an array of parameters
[{\tt 123}, {\tt esi}, {\tt edi}]. The {\tt peepgen\_code()} function will
then patch these parameters in the assembly fragment of {\tt rule450} to
obtain the translated binary code.

The {\tt peepgen\_code()} function is automatically constructed using
the {\tt peepgen} program. At compile time, the {\tt peepgen}
program (from {\tt peepgen\.c}) parses {\tt peep.tab} to generate a set
of labels (e.g., {\tt rule450}), their associated assembly fragments, and
the code to patch these assembly fragments with the appropriate parameters.
Notice that the Makefile invokes the {\tt peepgen} utility to generate
three header files: {\tt peepgen\_defs.h} (which declares and
defines the labels), {\tt peepgen\_entries.h} (which contain the peephole
table entries in a form that can be used to match the source instruction
template and obtain the corresponding assignments
using the {\tt templ\_matches\_insns()} function), and {\tt peepgen\_gencode.h}
(which contains the C code to patch the translated assembly fragments with
their parameters, in the {\tt peepgen\_code()} function).

\footnotesize{\small
\bibliographystyle{abbrv}
\bibliography{manual}}
\end{document}
